
%% LyX 2.3.6.2 created this file.  For more info, see http://www.lyx.org/.
%% Do not edit unless you really know what you are doing.
\documentclass[english,letterpaper,aps,prl,twocolumn]{revtex4}
\usepackage{lmodern}
\usepackage[T1]{fontenc}
\usepackage[latin1]{inputenc}
\setcounter{secnumdepth}{3}
\usepackage{amsmath}
\usepackage{graphicx}

\makeatletter
%%%%%%%%%%%%%%%%%%%%%%%%%%%%%% Textclass specific LaTeX commands.
\@ifundefined{textcolor}{}
{%
 \definecolor{BLACK}{gray}{0}f
 \definecolor{WHITE}{gray}{1}
 \definecolor{RED}{rgb}{1,0,0}
 \definecolor{GREEN}{rgb}{0,1,0}
 \definecolor{BLUE}{rgb}{0,0,1}
 \definecolor{CYAN}{cmyk}{1,0,0,0}
 \definecolor{MAGENTA}{cmyk}{0,1,0,0}
 \definecolor{YELLOW}{cmyk}{0,0,1,0}
}

%%%%%%%%%%%%%%%%%%%%%%%%%%%%%% User specified LaTeX commands.
%% LyX 1.5.3 created this file.  For more info, see http://www.lyx.org/.
%% Do not edit unless you really know what you are doing.

\usepackage{lmodern}






\makeatletter

%%%%%%%%%%%%%%%%%%%%%%%%%%%%%% LyX specific LaTeX commands.
%% Bold symbol macro for standard LaTeX users



%%%%%%%%%%%%%%%%%%%%%%%%%%%%%% User specified LaTeX commands.
%% LyX 1.5.3 created this file.  For more info, see http://www.lyx.org/.
%% Do not edit unless you really know what you are doing.

\usepackage{lmodern}









\makeatletter
%%%%%%%%%%%%%%%%%%%%%%%%%%%%%% User specified LaTeX commands.
%% LyX 1.5.3 created this file.  For more info, see http://www.lyx.org/.
%% Do not edit unless you really know what you are doing.

\usepackage{lmodern}








\makeatletter

%%%%%%%%%%%%%%%%%%%%%%%%%%%%%% LyX specific LaTeX commands.
%% Bold symbol macro for standard LaTeX users



%%%%%%%%%%%%%%%%%%%%%%%%%%%%%% User specified LaTeX commands.
%% LyX 1.5.3 created this file.  For more info, see http://www.lyx.org/.
%% Do not edit unless you really know what you are doing.






\makeatletter

%%%%%%%%%%%%%%%%%%%%%%%%%%%%%% LyX specific LaTeX commands.
%% Bold symbol macro for standard LaTeX users



%%%%%%%%%%%%%%%%%%%%%%%%%%%%%% User specified LaTeX commands.

%\documentclass[aps,prb,twocolumn,showpacs,preprintnumbers,amsmath,amssymb]{revtex4}

\usepackage{epsfig}





% Include figure files
\usepackage{dcolumn}




% Align table columns on decimal point
\usepackage{bm}




% bold math
\usepackage{bbm}





\usepackage{wasysym}





\usepackage{marvosym}





%\usepackage{times}


%\usepackage{epstopdf}


\newcommand{\la}{\langle}
\newcommand{\ra}{\rangle}

\newcommand{\revtex}{REV\TeX\ }
\newcommand{\classoption}[1]{\texttt{#1}}
\newcommand{\macro}[1]{\texttt{\textbackslash#1}}
\newcommand{\m}[1]{\macro{#1}}
\newcommand{\env}[1]{\texttt{#1}}
%\setlength{\textheight}{9.5in}
%\documentclass[aps,pra,groupedaddress,showpacs,preprint]{revtex4}

\usepackage{subfloat}% need for subequations
% need for figures
% useful for program listings
\usepackage{color}% use if color is used in text
\usepackage{subfigure}% use for side-by-side figures


\tolerance = 10000

%\usepackage{textcomp}
%\usepackage{graphicx}
%\usepackage{amsmath}
\usepackage{blindtext}

\newlength{\textwidthm}

\setlength{\textwidthm}{\columnwidth}

\addtolength{\textwidthm}{-\parindent}

\addtolength{\textwidthm}{-\parindent}


\makeatother


\makeatother

\usepackage{tikz}


\makeatother


\makeatother

\makeatother

\usepackage{babel}
\usepackage{pgffor}
\usepackage{pdfpages}
\begin{document}
\title{Strongly Interacting Phases in Twisted Bilayer Graphene at the Magic
Angle}
\author{Khagendra Adhikari$^{1}$, Kangjun Seo$^{2}$, K. S. D. Beach$^{3}$
and Bruno Uchoa$^{1}$}
\email{uchoa@ou.edu}

\affiliation{$^{1}$Department of Physics and Astronomy, University of Oklahoma,
Norman, Oklahoma 73069, USA}
\affiliation{$^{2}$School of Electrical and Computer Engineering, University of
Oklahoma, Tulsa, Oklahoma 74135, USA}
\affiliation{$^{3}$Department of Physics and Astronomy, The University of Mississippi,
University, Mississippi 38677, USA}

\date{\today}


\date{\today}
\begin{abstract}
Twisted bilayer graphene near the magic angle is known to have a
cascade of insulating phases at integer filling factors of the low-energy bands.
In this Letter we address the nature of these phases
through an unrestricted, self-consistent Hartree-Fock calculation on
the lattice that accounts for \emph{all} electronic bands. Using numerically
unbiased methods, we show that Coulomb interactions screened only
by metallic gates produce ferromagnetic insulating states at integer
fillings $\nu\in[-4,4]$ with maximal spin polarization $M_{\text{FM}}=4-|\nu|$.
With the exception of the $\nu=0,-2$ states, all other integer fillings
have insulating phases with additional sublattice symmetry breaking
and antiferromagnetism in the \emph{remote} bands. Valley polarization
is found away from half filling. Odd filling factors $|\nu|=1,3$
have anomalous quantum Hall states with Chern number $|\mathcal{C}|=1$,
whereas the $|\nu|=3$ states show strong particle-hole \emph{asymmetry}
in the small-gap regime. We map the
metal-insulator transitions of these phases as a function of the background
dielectric constant. 
\end{abstract}
\maketitle
{\it Introduction}.---When two graphene sheets are twisted by a
small angle, dubbed the ``magic angle,'' their low energy bands reconstruct
into flat bands \cite{Santos,Bistrizer}. A cascade of insulating
phases has been recently observed in twisted bilayer graphene (TBG)
near the magic angle at integer filling fractions of the flat bands
\cite{Cao1,Lu,Zondiner,Morissette,Park}, some of which are in proximity
to low-temperature superconducting phases \cite{Cao0,Yankowitz,Stepanov,Jaoui,Saito}.
Magic angle TBGs are considered strongly correlated electron systems,
with evidence of a strange metal phase at finite temperature \cite{Polshyn,Cao3}
and of low-energy electronic collective modes that are strongly coupled
to the charge carriers \cite{Saito-2}. Because of the quantum geometry
of the flat bands and the degeneracy between spin, valley, and sublattice
degrees of freedom in graphene \cite{Kotov}, interactions in combination
with magnetic field, strain and substrate effects can drive various
strongly interacting phases. Recent experiments observed evidence
of quantum anomalous Hall (AQH) states \cite{Spanton,Sharpe,Serlin,Wu},
charge order \cite{Jiang}, ferromagnetism \cite{Sharpe}, and Kekule
intervalley coherent states \cite{Nuckolls,Wagner,Kwan} in the presence
of strain. 

The precise nature of intrinsic correlated insulating phases at zero field,
in the absence of strain or substrate effects, is not entirely clear.
At the magic angle $\theta\approx1.1^{\circ}$, the Moir\'{e} unit cell
with size $L\sim a/[2\sin(\theta/2)]\approx12$ nm, where $a$ is
the graphene lattice constant, has $\sim10^{4}$ lattice sites, making
\emph{ab initio} methods prohibitively expensive. Previous numerical
studies based on Hartree-Fock \cite{Bultinick,M Xie,F Xie,Liu,Cea,Liu-1},
exact diagonalization \cite{Potasz,F Xie 2,Reppelin}, quantum Monte
Carlo \cite{Liao,Hofmann}, and density matrix renormalization group
\cite{Soejima,Kang} relied on effective low-energy models in the
continuum approximation, or else in effective lattice models derived
either with the extraction of Wannier orbitals from the low-energy
flat bands \cite{Kang-1,Koshino,Kang3,Seo,Breio} or with the introduction
of an energy cutoff in the remote bands \cite{Gonzalez}. At length
scales shorter than the size of the Moir\'{e} unit cell $L$, Coulomb
interactions can be one order of magnitude larger than the energy
gap separating the low-energy bands from the remote ones. A more conclusive
determination of the ground state needs to account for virtual processes
that connect the two sets of bands and may require treating the remote
bands on an equal footing with the flat ones. 

In this Letter we perform an unrestricted self-consistent Hartree-Fock
(HF) calculation on the lattice with Coulomb interactions screened
only by metallic gates. The calculation fully accounts for \emph{all}
remote bands in the emergence of many-body states. We find that the
insulating ground states of integer fillings $\nu\in[-3,3]$ are ferromagnetic,
with a total magnetic moment $M_{\text{FM}}=4-|\nu|$ induced by the
spin polarization of the low energy bands. Away from half-filling
($\nu=0$) and quarter filling $(\nu=-2)$ we identified antiferromagnetism
(AF) in the remote bands, which appears with sublattice polarization
(SLP). Valley polarization (VP) is found only away from half-filling.
Our results also indicate that odd filling factors $|\nu|=1,3$ have
quantum anomalous Hall (QAH) states with Chern number $|\mathcal{C}|$=1,
whereas even filling factors $|\nu|=0,2$ have zero net Chern number. 

Those insulating states have a metal-insulator transition (MIT) at
critical values of the effective dielectric constant $\varepsilon$,
as shown in Fig.~1. The amplitudes of the many-body gaps for $|\nu|=3$
are strongly particle-hole \emph{asymmetric} in the moderate- and small-gap
 regimes, in agreement with experiments \cite{Zondiner,Morissette}.
We conjecture that most experimental samples are in the vicinity of
the $\nu=0$ state MIT. In this regime, we find that the $\nu=\pm2,3$
states have more prominent gaps of a few meV, whereas the other filling
factors $\nu=-3,\pm1$ are either metallic or have very small many-body
gaps. 

\begin{figure}
\includegraphics[width=0.95\columnwidth]{Fig1.pdf}

\caption{{\small{}a) Phase diagram of strongly interacting phases in magic
angle TBG at integer fillings $\nu\in[-3,3]$ of the flat bands for
different values of the background dielectric constant $\varepsilon$.
Size of the circles and color indicate the amplitude of the many-body
gap $\Delta$. Small dots are metallic states (zero gap), indicating
the presence of a metal-insulator transition for a critical value
of $\varepsilon$. b) Scaling of the many-body gap $\Delta$ vs $1/\varepsilon$
for different integer fillings. }}
\end{figure}

{\it Hamiltonian}.---We begin with the real space lattice Hamiltonian
of TBG, $\mathcal{H}=\mathcal{H}_{0}+\mathcal{H}_{\text{C}},$ where
\begin{equation}
\mathcal{H}_{0}=\sum_{\alpha\beta}\sum_{ij}h_{\alpha\beta}(\mathbf{R}_{i}-\mathbf{R}_{j})d_{\alpha,i,\sigma}^{\dagger}d_{\beta,j,\sigma}\label{eq:H0}
\end{equation}
is the hopping term, with $d_{\alpha,i,\sigma}$ the annihilation
operator of an electron with spin $\sigma=\uparrow,\downarrow$ on
sublattice $\alpha$ of unit cell $i$, and $h_{\alpha\beta}(\mathbf{R}_{i}-\mathbf{R}_{j})$
is the corresponding tight-binding matrix element between sublattices
$\alpha$ and $\beta$ located in unit cells centered at $\mathbf{R}_{i}$
and $\mathbf{R}_{j}$ respectively. The second term describes a screened
Coulomb interaction between any two lattice sites, 
\begin{equation}
\mathcal{H}_{\text{C}}=\frac{1}{2}\sum_{\alpha\beta}\sum_{ij}\hat{n}_{\alpha,i}V_{\alpha\beta}(\mathbf{R}_{i}-\mathbf{R}_{j})\hat{n}_{\beta,j},\label{eq:Hc}
\end{equation}
where $\hat{n}_{\alpha,i}=\sum_{\sigma}d_{\alpha,i,\sigma}^{\dagger}d_{\alpha,i,\sigma}$
is the local density operator and $V_{\alpha\beta}(\mathbf{R})=e^{2}/(\varepsilon\xi)\sum_{m=-\infty}^{\infty}(-1)^{m}[(|\boldsymbol{\tau}_{\alpha}-\boldsymbol{\mathbf{\tau}}_{\beta}+\mathbf{R}|/\xi)^{2}+m^{2}]^{-\frac{1}{2}}$
is the screened form of the interaction in the presence of symmetric
gates located at the top and bottom of the TBG heterostructure
\cite{Throckmorton,Bernevig2}, with $e$ the electron charge and
$\boldsymbol{\tau}_{\alpha}$ the position of a site in sublattice
$\alpha$ measured from the center of its unit cell. $\xi\approx10$\,nm
is the distance between the plates of the metallic gates in most experiments
\cite{Stepanov,Saito}, and $\varepsilon$ is the background dielectric
constant, which we treat as a free adjustable parameter. The value
of the dielectric constant of graphene encapsulated in hexagonal boron
nitride (hBN) is $\varepsilon\approx6$, although polarization effects
due to remote bands in TBG could effectively make it several times
larger \cite{Potasz,Gonzalez}. We regularize the on-site Coulomb
interaction by choosing the onsite Hubbard term $V_{\alpha\alpha}(\mathbf{R}=0)=12.4/\varepsilon\,$eV
of single layer graphene \cite{Wehling} and then smoothly interpolating
it with Eq.~\eqref{eq:Hc} through $V_{\alpha\beta}(\mathbf{R})\approx1.438/[\varepsilon(0.116+|\boldsymbol{\tau}_{\alpha}-\boldsymbol{\mathbf{\tau}}_{\beta}+\mathbf{R}|)]$
eV \cite{Radamaker}. Although this choice is not unique, the results
do not depend on the details of the regularization. 

Strain effects can favor ground states that break the translational
symmetry of the Moir\'{e} pattern \cite{Nuckolls,Wagner,Kwan}. To the
best of our knowledge, no broken translation symmetry has been observed
to date in TBG without strain \cite{Xie,Kerelsky}. We assume the
absence of strain effects and rewrite the Hamiltonian in momentum
representation, 
\begin{equation}
\mathcal{H}=\sum_{\alpha\beta}\sum_{\mathbf{k},\sigma}h_{\alpha\beta}(\mathbf{k})d_{\alpha,\mathbf{k},\sigma}^{\dagger}d_{\beta,\mathbf{k},\sigma}+\mathcal{H}_{\text{C}},\label{eq:H0-1}
\end{equation}
where
\begin{equation}
\mathcal{H}_{\text{C}}=\frac{1}{2}\sum_{\alpha\beta}\sum_{\mathbf{q}}\hat{n}_{\alpha}(\mathbf{q})V_{\alpha\beta}(\mathbf{q})\hat{n}_{\beta}(-\mathbf{q}),\label{eq:HC2}
\end{equation}
with $V_{\alpha\beta}(\mathbf{q})$ being the Fourier transform of the Coulomb
interaction in Eq.~\eqref{eq:Hc}. 

\begin{figure*}
\includegraphics[width=0.95\textwidth]{Fig2.pdf}

\caption{{\small{}Band structure of the low energy spin polarized bands for
integer filling factors. Top row from left to right: $\nu=3,2,1,0$
for $\varepsilon=6$. Bottom row (left to right): $\nu=-3,-2,-1$
for $\varepsilon=6$ and $\nu=0$ for $\varepsilon=30$ (metallic
state). The gray shadow indicates the size of the many-body gap. The red lines represent spin $\sigma$ and the blue ones $-\sigma$.
Red (blue) bands are topological for $\nu<0$ ($\nu>0$) and trivial
otherwise. Odd and even filling factors have a net Chern number $\mathcal{C}=1$
and zero, respectively. 
At $\varepsilon=6$ all integer filling factors have insulating states. 
The insulating state at  $\nu=0$
is ferromagnetic, with zero valley polarization. All non-zero filling
states are spin and valley polarized. The $\nu=0$ state has a MIT
in the range $24<\varepsilon<30$ (see text). }}
\end{figure*}

The size of the Moir\'{e} unit cell of commensurate structures is set
by two integers $s$ and $m$, which determine the twist angle via $\cos\theta=1-s^{2}/[2(3m^{2}+3ms+s^{2})]$.
Near the magic angle $\theta\approx1.085$ ($s=1$, $m=30$) the unit
cell has $N_{b}=11164$ lattice sites, and $\alpha=1,\ldots,N_{b}$.
The tight-binding hopping parameters are calculated from the standard
parametrization $t(\mathbf{r})=\cos^{2}\theta_{z}V_{\sigma}(\mathbf{r})+\sin^{2}\theta_{z}V_{\pi}(\mathbf{r})$,
where $\cos\theta_{z}=d/|\mathbf{r}|$, with $d$ is the vertical
distance between sites. $V_{\sigma}(\mathbf{r})$ and $V_{\pi}(\mathbf{r})$
are Slater-Koster functions parametrized by {\it ab initio} calculations
\cite{Note,Nam}. We define $h_{\alpha\beta}(\mathbf{R}_{i}-\mathbf{R}_{j})\equiv t(\mathbf{r}_{\alpha}-\mathbf{r}_{\beta})-\mu\delta_{\alpha\beta}\delta_{ij}$,
with $\mathbf{r}_{\alpha}=\tau_{\alpha}+\mathbf{R}_{i}$ the position
of the lattice sites and $\mu$ the chemical potential. We account
for lattice corrugation effects by allowing the layers to relax in
the $z$ direction. The relaxed tight-binding band structure of TBG
has eight low energy bands with a band width of $\sim 6$\,meV separated
from the remote bands by an energy gap of $20$ meV \cite{Nam,Uchida,Note2}.
At the Hartree level, charging effects are expected to significantly
reconstruct the low energy bands \cite{Guinea}. 

{\it Hartree-Fock calculation}.---We proceed to identify the many-body
ground states of the problem through an unrestricted, self-consistent
HF calculation that accounts for \emph{all} the electronic bands. A HF decomposition of the Hamiltonian reduces the problem to one of diagonalizing the effective Hamiltonian 
\begin{equation}
\mathcal{H}_{\text{HF}}=\sum_{\mathbf{k},\sigma}\sum_{\alpha\beta}\bar{h}_{\alpha\beta}(\mathbf{k},\sigma)d_{\alpha,\mathbf{k},\sigma}^{\dagger}d_{\beta,\mathbf{k},\sigma}.\label{eq:H_HF}
\end{equation}
Here $\bar{h}_{\alpha\beta}(\mathbf{k},\sigma)=h_{\alpha\beta}(\mathbf{k})+h_{\alpha\beta}^{\text{H}}(\mathbf{k},\sigma)+h_{\alpha\beta}^{\text{F}}(\mathbf{k},\sigma)$
is the renormalized matrix elements due to both Hartree, 
\begin{equation}
h_{\alpha\beta}^{\text{H}}(\mathbf{k},\sigma)=\delta_{\alpha\beta}\sum_{\gamma,\mathbf{k}^{\prime},\sigma^{\prime}}V_{\beta\gamma}(0)\rho_{\gamma\gamma}(\mathbf{k}^{\prime},\sigma^{\prime}),\label{eq:H_H}
\end{equation}
and Fock contributions, 
\begin{equation}
h_{\alpha\beta}^{\text{F}}(\mathbf{k},\sigma)=-\sum_{\mathbf{k}^{\prime}}V_{\alpha\beta}(\mathbf{k}-\mathbf{k}^{\prime})\rho_{\alpha\beta}(\mathbf{k}^{\prime},\sigma).\label{eq:H_F}
\end{equation}
Each contribution can be cast in terms of the $N_{b}\times N_{b}$
zero-temperature density matrix for a given momentum and spin,
\begin{equation}
\rho_{\alpha\beta}(\mathbf{k},\sigma)=\sum_{n}^{\text{occupied}}\phi_{\alpha,\mathbf{k}}^{(n)}(\sigma)\phi_{\beta,\mathbf{k}}^{(n)*}(\sigma),\label{eq:rho}
\end{equation}
which is defined as a sum over all occupied bands, with $n=1,\ldots,N_{b}$.
The $N_{b}$-component spinors $\phi_{\alpha,\mathbf{k}}^{(n)}(\sigma)$
describe the exact eigenvectors of Eq.~\eqref{eq:H_HF}, which need to
be calculated self-consistently while enforcing a fixed total number of
particles per unit cell. 

The ground state at the HF level is found by self-consistently calculating
the real space density matrix, $\hat{\rho}_{\alpha\beta}(\mathbf{R},\sigma)$,
from a $9\times9$ mesh in momentum space including the $\Gamma$
point. Each self-consistent loop renormalizes the chemical potential
of the previous step to ensure conservation of the total number of
particles. All densities are measured away from the neutrality point.
We subtract the density matrix from a reference density matrix corresponding
to a uniform background of charge. The initial density matrix is 
extracted from the relaxed non-interacting tight binding Hamiltonian
and self-consistently iterated until convergence. We implement a strong
convergence criteria for the density matrix, where the root mean square
of the difference in density matrices with the previous iteration
is less than the tolerance, $\delta\hat{\rho}_{\alpha\beta}(0,\sigma)<10^{-8}$.
In order to probe for many-body instabilities, we seed the initial
density matrix with random noise, allowing for all symmetries (with
the exception of translational symmetry) to be broken. After convergence,
we then perform numerical measurements with the resulting density
matrix to identify the ground states at different filling factors.
This approach provides an unbiased method to probe for many-body instabilities
at the HF level. 

{\it Ferromagnetic ground states}.---In Fig.~2 we show the reconstructed
low energy bands of magic angle TBG for integer filling factors $\nu\in[-3,3]$
for $\varepsilon=6$. At this value of $\varepsilon$ we find that
all integer fillings correspond to insulating states. Interactions
increase the bandwidth of the low-energy flat bands by a factor of
$\sim3$, while splitting their spin degeneracy. The lines in red
(blue) are spin $\sigma$ ($-\sigma$) bands, which are fully spin
polarized, with large many-body gaps $\Delta$ (see Fig.~1a). For
$\nu=-3$ the ground state is a ferromagnetic insulator with a single
occupied flat band with spin $\sigma$, as depicted in Fig.~2a.
At higher non-positive integer fillings $\nu\leq0$ (Fig.~2b--d)
the spin polarization $M_{\text{FM}}$ increases in integer increments
and is maximal at half filling ($\nu=0$), where the ground state
has four occupied mini-bands with the same spin $\sigma$. At
positive integer fillings ($\nu>0$) additional mini-bands bands with
spin $-\sigma$ are occupied (Fig.~2e--h). We find that the the
system has maximum total spin polarization $M_{\text{FM}}=4-|\nu|$
for $\nu\in[-4,4]$. This behavior is observed in all insulating states
at higher values of $\varepsilon$. For $\varepsilon=12$ and $24$
the many-body gaps progressively decrease while the low-energy bands
become flatter, suggesting a series of filling-factor-dependent MITs
\cite{Note2-1}. For instance, at $\varepsilon=24$ the $\nu=-3$
state is metallic whereas the $\nu=0$ state has a MIT below $\varepsilon=30$
(see Fig.~2). The scaling of $\Delta$ with $1/\varepsilon$ for all
integer fillings is shown in Fig.~1b. 

\begin{figure}[t]
\includegraphics[width=0.92\columnwidth]{Fig3.pdf}

\caption{{\small{}Charge and spin distribution on the moire unit cell in magic
angle TBG for $\varepsilon=6$ and $\nu=3$. a) Charge per site $\hat{\rho}_{\alpha\alpha}(0,\sigma)$
of the low energy flat bands, concentrated near $AA$ regions, at
the corners on the unit cell (orange diamond), shown in b). The zoomed
charge pattern shows sublattice symmetry breaking (SLP order). c)
Spin polarization favors ferromagnetism, with amplitude modulation
at the sublattice scale near $AA$ sites (red diamond), as shown in
d).}}
\end{figure}

The ferromagnetic order parameter can be defined directly from the
real-space density matrix as $M_{\text{FM}}=\sum_{\alpha}s_{\alpha}$,
with $s_{\alpha}=\hat{\rho}_{\alpha\alpha}(0,\sigma)-\hat{\rho}_{\alpha\alpha}(0,-\sigma)$
the local spin polarization summed over all lattice sites in the Moir\'{e}
unit cell. In Fig.~3c we show the values of the real-space magnetization
extracted from the density matrix for $\nu=3$ ($\varepsilon=6$).
Only the flat bands contribute to the magnetization $M_{\text{FM}}$
and to the ensuing ferromagnetic order. The ferromagnetic states are
concentrated in \emph{AA} site regions, where the charge density of
the flat bands is also concentrated (see Fig.~3a). We summarize in
Fig.~4a the explicit measurement of $M_{\text{FM}}$ through the density
matrix for different integer fillings. 

With the exception of half and quarter filling ($\nu=0,-2$), we find
SLP in all other insulating states. The order parameter for SLP breaks
the symmetry between sublattices $A$ and $B$ in each monolayer and
is defined as $m_{\text{SLP}}=\sum_{\sigma,\alpha\in A}\hat{\rho}_{\alpha\alpha}(0,\sigma)-\sum_{\sigma,\alpha\in B}\hat{\rho}_{\alpha\alpha}(0,\sigma)$.
In Fig.~4b we show the contribution to SLP from both the low-energy and
remote bands. We find that SLP is accompanied by a sublattice modulation
of the ferromagnetism in the low-energy bands, as shown in Fig.~3d, and
by a pattern of \emph{staggered} magnetization in the \emph{remote}
bands, indicating AF order. In Fig.~4c we display the AF order parameter
$M_{\text{AF}}=\sum_{\alpha\in A}s_{\alpha}-\sum_{\alpha\in B}s_{\alpha}$
for different integer fillings. Both $M_{\text{AF}}$ and $m_{\text{SLP}}$
vanish at $\nu=0,-2$ and closely follow each other at other integer
filling factors, as shown in Fig.~4b, c. On the other hand, the presence
of weak SLP and AF order at $\nu=2$ suggests that particle-hole symmetry
is broken. 

{\it Topological states}.---We next calculate the Chern number
of the flat bands through the eigenstates extracted from the self-consistent
density matrix, $\mathcal{C}_{n}(\sigma)=-(i/2\pi)\int_{\text{BZ}}\text{d}^{2}k\,\hat{n}\cdot[\nabla_{\mathbf{k}}\times\langle\phi_{\mathbf{k}}^{(n)}(\sigma)|\nabla_{\mathbf{k}}|\phi_{\mathbf{k}}^{(n)}(\sigma)\rangle],$
with the integral set over the 2D Brillouin zone oriented in the $\hat{n}$-axis.
In the panels of Fig.~2 we depict the Chern number of the topological
flat bands with their corresponding spin polarization. The bands in
red (blue) are topological for $\nu<0$ ($\nu>0$) and trivial otherwise.
We find that the total Chern number of the occupied minibands with
$|\nu|=1,3$ is $|\mathcal{C}|=1$ for all values of $\varepsilon$
with insulating states. At integer fillings $|\nu|=0,2$ the occupied
mini-bands have zero net Chern number. We verify this result by doing
numerical measurements of the QAH order parameter through the real-space density matrix,
\begin{equation}
m_{\text{QAH}}=\frac{1}{3\sqrt{3}}\,\text{Im}\!\!\sum_{\langle\langle\alpha,\beta\rangle\rangle}\eta_{\alpha\beta}\hat{\rho}{}_{\alpha\beta}(0,\sigma),\label{eq:Maqh}
\end{equation}
which probes for chiral loop currents among next-nearest-neighbor
sites $\langle\langle\alpha,\beta\rangle\rangle$, where $\eta_{\alpha\beta}=\pm1$
for clockwise or counterclockwise hopping in the honeycomb plaquette
of each monolayer \cite{Haldane}. As shown in Fig.~4d, $m_{\text{QAH}}=0$
at even filling factors and is finite at odd fillings, in agreement
with the calculated Chern numbers depicted in Fig.~4e. We note that
the $\nu=-3$ state for $\varepsilon=24$ is a spin polarized metal
with a finite optical gap at the $\Gamma$ point \cite{Note2-1}.
Although we observe bulk loop currents in this state ($m_{\text{QAH}}\neq0$),
it does not translate into the emergence of chiral topological edge modes and the Chern
number is not well defined. 

\begin{figure}[b]
\includegraphics[width=1\columnwidth]{Fig4.pdf}

\caption{{\small{}Magic angle TBG many-body instabilities at $\varepsilon=6$
(red dots), 12 (blue diamonds) and 24 (orange triangles) for integer
fillings factors $\nu\in[-3,3]$. a) Spin polarization (FM), b) sublattice
polarization (SLP), c) antiferromagnetism (AF), d) quantum anomalous
Hall (QAH), e) Chern number (}\emph{\small{}$\mathcal{C}$}{\small{})
and f) valley polarization (VP). QAH states with Chern number $\mathcal{C}=1$
appear at odd factors $|\nu|=1,3$ (see text). }}
\end{figure}

{\it Valley polarization}.---The valleys of the graphene monolayers
are not good quantum numbers in lattice models. Nevertheless, they
are emergent quantum numbers that set the overall degeneracy of the
flat bands in the non-interacting case \cite{Pereira,Dou}. To measure
valley polarization (VP) effects we use the order parameter
\begin{equation}
m_{\text{VP}}=\frac{1}{3\sqrt{3}}\:\text{Im}\negmedspace\sum_{\langle\langle\alpha,\beta\rangle\rangle}\delta_{\alpha}\eta_{\alpha\beta}\hat{\rho}{}_{\alpha\beta}(0,\sigma),\label{eq:mV}
\end{equation}
with $\delta_{\alpha}=\pm1$ for $\alpha\in A$ or $B$, which sums
over the difference between the loop currents in sublattices $A$
and $B$ in each monolayer \cite{Ramires}. In Fig.~4f we show the
total valley polarization (VP) calculated for integer fillings. The
behavior of $m_{\text{VP}}$ is not monotonic with the strength of
interactions, unlike the other order parameters. For all values of
$\varepsilon$ we find that VP and SLP are Pauli blocked at half filling,
$m_{\text{VP}}=m_{\text{SLP}}=0$, due to the $\nu=0$ state being
maximally spin polarized ($M_{\text{FM}}=4$), exhausting the occupation
of the flat bands with the same spin. 

{\it Discussion}.---Exact diagonalization results \cite{Potasz}
and exact calculations in the chiral limit of the continuum model
\cite{Bernevig} indicate that unrestricted HF can faithfully capture
the ground state of magic angle TBG at integer fillings, where correlation
effects are minimal, and interactions are mostly dominated by exchange.
Strong interactions, nevertheless, can mix very different energy and
length scales. Previous attempts to incorporate remote band effects
were mostly based on self-energy corrections to the flat bands in
the form of interaction assisted hopping terms \cite{Potasz,Liao,Kang3,Guinea}.
Because of the large degeneracy of the flat bands, however, where spin,
sublattice and valley polarized ground states can either compete or
mutually assist, we posit that the determination of the ground state
requires a lattice calculation that treats all remote bands on an equal
footing with the flat ones using unbiased numerical methods. 

Calculations based on lo-energy models predicted a variety of different
insulating ground states for intrinsic magic angle TBG, including
SU(4) magnetism at $\nu=-2$ \cite{Kang3,Seo}, nematic phases \cite{Liu},
quantum valley Hall \cite{Liao,Breio} and intervalley coherent insulators
\cite{Bultinick,Hofmann}. Our unrestricted, HF results on the lattice,
including all bands, show that all integer filling factors $\nu\in[-3,3]$
support insulating ferromagnetic states that are maximally spin polarized.
We did not observe ground states that spontaneously lower the rotational
point group symmetry of the crystal. Away from half filling, the valleys
and sublattices polarize, in agreement with exact diagonalization
results for $|\nu|\in[2,3]$ \cite{Potasz}. We show that SLP is accompanied
with the emergence of AF order in the remote bands, and with sublattice
modulation of the ferromagnetically ordered state in the flat bands.
Recent electron spin resonance transport measurements have found evidence
of magnetism and AF order in magic angle TBG at $\nu=\pm2,3$ \cite{Morissette}.
We also find that QAH states with Chern number $\mathcal{C}=1$ emerge
at odd integer filling factors, in agreement with zero-field transport
measurements in hexagonal-boron nitride non-aligned samples \cite{Stepanov-1}. 

Our numerical calculations also shed light into the MIT transition
of those insulating states. Transport experiments indicate the prevalence
of insulating states at $\nu=\pm2,\,3$, whereas the one at $\nu=0$
has been observed in some experiments \cite{Cao1} but not in others
\cite{Zondiner,Morissette}. According to the scaling of the many-body
gaps shown in Fig.~1b, the typical transport gap of a few meV for
$\nu=-2$ observed in most HBN encapsulated samples is consistent
with an effective dielectric constant $\varepsilon\sim25\pm5$. We
predict that solid insulating behaviors at $\nu=0$ and $\nu=\pm1$
are to be expected in suspended samples.

{\it Acknowledgments.}---We acknowledge V. N. Kotov and E. Khalaf
for helpful discussions. K.\ A.\ and B.\ U.\  acknowledge NSF grant DMR-2024864
for support. We also acknowledge OSCER and MCSR supercomputing centers
for support. 
\begin{thebibliography}{10}
\bibitem{Santos}J. M. B. Lopes dos Santos, N. M. Peres, A. H. Castro
Neto, Phys. Rev. Lett. \textbf{99}, 256802 (2007).

\bibitem{Bistrizer}R. Bistritzer and A. H. MacDonald,\emph{ }Proc.
Nat. Acad. Sci. U.S.A. \textbf{108}, 12233 (2011).

\bibitem{Cao1}Y. Cao, V. Fatemi, A. Demir, S. Fang, S. L. Tomarken,
J. Y. Luo, J. D. Sanchez-Yamagishi, K. Watanabe, T. Taniguchi, E.
Kaxiras, R. C. Ashoori, and P. Jarillo-Herrero, Nature \textbf{582,}\emph{
}80 (2018). 

\bibitem{Lu}X. Lu, P. Stepanov, W. Yang, M. Xie, M. Ali Aamir, I.
Das, C. Urgell, K. Watanabe, T. Taniguchi, G. Zhang, A. Bachtold,
A. H. MacDonald, D. K. Efetov, Nature \textbf{574}, 653 (2019).

\bibitem{Zondiner}U. Zondiner, A. Rozen, D. Rodan-Legrain, Y. Cao,
R. Queiroz, T. Taniguchi, K. Watanabe, Y. Oreg, F. von Oppen, Ady
Stern, E. Berg, P. Jarillo-Herrero, and S. Ilani, Nature \textbf{582},
203 (2020).

\bibitem{Park}J. M. Park, Y. Cao, K. Watanabe, T. Taniguchi and Pablo
Jarillo-Herrero, Nature \textbf{592}, 43(2021).

\bibitem{Morissette}E. Morissette, J.-X. Lin, D. Sun, L. Zhang, S.
Liu, D. Rhodes, K. Watanabe, T. Taniguchi, J. Hone, J. Pollanen, M.
S. Scheurer, M. Lilly, A. Mounce, Nat. Phys. https://doi.org/10.1038/s41567-023-02060-0
(2023). 

\bibitem{Cao0}Y. Cao, V. Fatemi, S. Fang, K. Watanabe, T. Taniguchi,
E. Kaxiras, and P. Jarillo-Herrero, Nature \textbf{582,}\emph{ }42
(2018). 

\bibitem{Yankowitz}M. Yankowitz, S. Chen, H. Polshyn, Y. Zhang, K.
Watanabe, T. Taniguchi, D. Graf , A. F. Young, C. R. Dean, Science
\textbf{363}, 1059 (2019).

\bibitem{Stepanov}P. Stepanov, I. Das, X. Lu, A. Fahimniya, K. Watanabe,
T. Taniguchi, F. H. L. Koppens, J. Lischner, L. Levitov, D. K. Efetov,
Nature \textbf{583} 375 (2020).

\bibitem{Jaoui}A. Jaoui, I. Das, Giorgio Di Battista Xiaobo Lu, K.
Watanabe, T. Taniguchi, Jaime Dez-Mrida, H. Ishizuka, L. Levitov
and D. K. Efetov, Nat. Physics \textbf{18}, 633 (2020). 

\bibitem{Saito}Y. Saito, J. Ge, K. Watanabe, T. Taniguchi and A.
F. Young, Nat. Phys. \textbf{16}, 926 (2020). 

\bibitem{Polshyn}H. Polshyn, M. Yankowitz, S. Chen, Y. Zhang, K.
Watanabe, T. Taniguchi, C. R. Dean and A. F. Young, Nat. Phys. \textbf{15},
1011 (2019). 

\bibitem{Cao3}Y. Cao, D. Chowdhury, D. R.-Legrain, O. Rubies-Bigorda,
K. Watanabe, T. Taniguchi, T. Senthil, and P. Jarillo-Herrero, Phys.
Rev. Lett. \textbf{124}, 076801 (2020).

\bibitem{Saito-2}Y. Saito, F. Yang, J. Ge, X. Liu, T. Taniguchi,
K.Watanabe, J. I. A. Li, E. Berg, and A. F. Young, Nature \textbf{592},
220 (2021).

\bibitem{Kotov}V. N. Kotov, B. Uchoa, V. M. Pereira, F. Guinea, and
A. H. Castro Neto, Rev. Mod. Phys. \textbf{84}, 1067 (2012). 

\bibitem{Spanton}E. M. Spanton, A. A. Zibrov, H. Zhou, T. Taniguchi,
K. Watanabe, M. P. Zaletel, and A. F. Young, Science \textbf{360},
62 (2018).

\bibitem{Sharpe}A. L. Sharpe, E. J. Fox, A. W. Barnard, J. Finney,
K. Watanabe, T. Taniguchi, M. A. Kastner, and D. Goldhaber-Gordon,
Science \textbf{365}, 605 (2019).

\bibitem{Serlin}M. Serlin, C. L. Tschirhart, H. Polshyn, Y. Zhang,
J. Zhu, K. Watanabe, T. Taniguchi, L. Balents, and A. F. Young, Science
\textbf{367}, 900 (2020).

\bibitem{Wu}S. Wu, Z. Zhang1, K. Watanabe, T. Taniguchi and E. Y.
Andrei, Nat. Materials \textbf{20}, 488 (2021).

\bibitem{Jiang}Y. Jiang, X. Lai, K. Watanabe, T. Taniguchi, K. Haule,
J. Mao, and Y. Andrei, Nature \textbf{573}, 91 (2019).

\bibitem{Nuckolls}K. P. Nuckolls, R. L. Lee, M. Oh, D. Wong1, T.
Soejima, J. P. Hong, D. C\u{a}lug\u{a}ru, J. Herzog-Arbeitman, B.
A. Bernevig1, K. Waanabe, T. Taniguchi, N. Regnault, M. P. Zaletel,
A. Yazdani, arXiv:2303.00024 (2023).

\bibitem{Wagner}G. Wagner, Y. H. Kwan, N. Bultinck, S. H. Simon,
and S. A. Parameswaran, Phys. Rev. Lett. \textbf{128}, 156401 (2022).

\bibitem{Kwan}Y. H. Kwan , G. Wagner, T. Soejima, M. P. Zaletel,
S. H. Simon, S. A. Parameswaran, and N. Bultinck, Phys. Rev. X \textbf{11},
041063 (2021).

\bibitem{Bultinick}N. Bultinck, E. Khalaf, S. Liu, S. Chatterjee,
A. Vishwanath, and M. P. Zaletel, Phys. Rev. X \textbf{10}, 031034
(2020).

\bibitem{M Xie}M. Xie and A. H. MacDonald, Phys. Rev. Lett. \textbf{124},
097601 (2020).

\bibitem{F Xie}F. Xie, J. Kang, B. Andrei Bernevig, O. Vafek, and
N. Regnault, Phys. Rev. B \textbf{107}, 075156 (2023).

\bibitem{Liu}S. Liu, E. Khalaf, J. Yeon Lee, and A. Vishwanath, Phys.
Rev. Rev. \textbf{3}, 013033 (2021).

\bibitem{Cea}T. Cea and F. Guinea, Phys. Rev. B \textbf{102}, 045107
(2020).

\bibitem{Liu-1}J. Liu and X. Dai, Phys. Rev. B \textbf{103}, 035427
(2021).

\bibitem{Potasz}P. Potasz, M. Xie and A. H. MacDonald, Phys. Rev.
Lett. \textbf{127}, 147203 (2021).

\bibitem{F Xie 2}F. Xie, A. Cowsik, Z.-D. Song, B. Lian, B. A. Bernevig,
and N. Regnault, Phys. Rev. B \textbf{103}, 205416 (2021).

\bibitem{Reppelin}C. Repellin , Z. Dong, Y.-H. Zhang, and T. Senthil,
Phys. Rev. Lett. \textbf{124}, 187601 (2020).

\bibitem{Liao}Y. Da Liao, J. Kang , C. N. Brei, X. Yan Xu, Han-Qing
Wu, B. M. Andersen, R. M. Fernandes, and Zi Yang Meng, Phys. Rev.
X \textbf{11}, 011014 (2021).

\bibitem{Hofmann}J. S. Hofmann , E. Khalaf, A. Vishwanath, E. Berg,
and J. Y. Lee, Phys. Rev. X \textbf{12}, 011061 (2022).

\bibitem{Soejima}T. Soejima, D. E. Parker, N. Bultinck, J. Hauschild,
and M. P. Zaletel, Phys. Rev. B \textbf{102}, 205111 (2020).

\bibitem{Kang}J. Kang, and O. Vafek, Phys. Rev. B \textbf{102}, 035161
(2020).

\bibitem{Kang-1}J. Kang, and O. Vafek, Phys. Rev. X \textbf{8}, 031088
(2018).

\bibitem{Koshino}M. Koshino, N. F. Q. Yuan, T. Koretsune, M. Ochi,
K. Kuroki, and L. Fu, Phys. Rev. X \textbf{8}, 031087 (2018).

\bibitem{Kang3}J. Kang, and O. Vafek, Phys. Rev. Lett. \textbf{122}
246401 (2019).

\bibitem{Seo}K. Seo, V. N. Kotov, and B. Uchoa, Phys. Rev. Lett.
\textbf{122} 246402 (2019).

\bibitem{Breio}C.N. Brei and B. M. Andersen, Phys. Rev. B \textbf{107},
165114 (2023). 

\bibitem{Gonzalez}J. Gonzlez and T. Stauber, Phys. Rev. B \textbf{104},
115110 (2021).

\bibitem{Throckmorton}R. E. Throckmorton and O. Vafek, Phys. Rev.
B 86, 115447 (2012).

\bibitem{Bernevig2}B. Andrei Bernevig, Zhi-Da Song, Nicolas Regnault,
and Biao Lian, Phys. Rev. B \textbf{103}, 205413 (2021).

\bibitem{Wehling}T. O. Wehling, E. Sasioglu, C. Friedrich, A. I.
Lichtenstein, M. I. Katsnelson, and S. Blugel, Phys. Rev. Lett. \textbf{106},
236805 (2011).

\bibitem{Radamaker}L. Rademaker, D. A. Abanin,1 and P. Mellado, Phys.
Rev. B \textbf{100}, 205114 (2019).

\bibitem{Note}We use $V_{\sigma}(\mathbf{r})=V_{\sigma}^{0}\text{exp}\left[-(|\mathbf{r}|-d_{0})/\lambda\right]$
and $V_{\pi}(\mathbf{r})=V_{\pi}^{0}\text{exp}\left[-(\delta r-a/\sqrt{3})/\lambda\right]$,
with $V_{\sigma}^{0}=-0.48$eV, $V_{\pi}^{0}=0.27$eV, and $\lambda=0.4525\text{}$.
$d_{0}=3.35\text{}$ is the unrelaxed interlayer distance and $a=2.42\text{}$
the in-plane lattice constant. See Ref. \cite{Nam}.

\bibitem{Xie}Y. Xie, B. Lian, B. Jck, X. Liu, C.-L. Chiu, K. Watanabe,
T. Taniguchi, B. Andrei Bernevig, and A. Yazdani, Nature \textbf{572},
101 (2019).

\bibitem{Kerelsky}A. Kerelsky, L. J. McGilly, D. M. Kennes, L. Xian,
M. Yankowitz, S. Chen, K. Watanabe, T. Taniguchi, J. Hone, Cory Dean,
A. Rubio, and A. N. Pasupathy, Nature \textbf{572}, 95 (2019).

\bibitem{Nam}N. N. T. Nam and M. Koshino, Phys. Rev. B \textbf{96},
075311 (2017).

\bibitem{Uchida}K. Uchida, S. Furuya, J.-I. Iwata, and A. Oshiyama,
Phys. Rev. B \textbf{90}, 155451 (2014).

\bibitem{Note2}For details on the tight-binding band structure with
lattice relaxation effects, see supplemental materials. 

\bibitem{Guinea}F. Guinea, and N. R. Walet, Proc. Nat. Acad. Sci.
U.S.A. \textbf{115}, 131714 (2018).

\bibitem{Note2-1}For more details about the interacting band structure
and their MITs for larger values of $\varepsilon$, see supplemental
materials. 

\bibitem{Pereira}V. M. Pereira, V. N. Kotov, and A. H. Castro Neto,
Phys. Rev. B \textbf{78}, 085101 (2008). 

\bibitem{Dou}X. Dou, V. N. Kotov, B. Uchoa, Sci. Rep. \textbf{6},
31737 (2016). 

\bibitem{Haldane}F. D. M. Haldane, Phys. Rev. Lett. \textbf{61},
2015 (1988). 

\bibitem{Ramires}A. Ramires and J. L. Lado, Phys. Rev. B \textbf{99},
245118 (2019).

\bibitem{Bernevig}B. Lian, Z.-D. Song, N. Regnault, D. K. Efetov,
A. Yazdani, and B. Andrei Bernevig, Phys. Rev. B \textbf{103}, 205414
(2021). 

\bibitem{Stepanov-1}P. Stepanov, M. Xie, T. Taniguchi, K. Watanabe,
X. Lu, A. H. MacDonald, B. Andrei Bernevig, and D. K. Efetov, Phys.
Rev. Lett. \textbf{127}, 197701 (2021).
\end{thebibliography}
\newpage
\foreach \x in {1,...,8}
{%
\clearpage
\includepdf[pages={\x,{}}]{SM_TBG}
}
\end{document}
