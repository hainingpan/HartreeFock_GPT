\documentclass{article}
\usepackage[utf8]{inputenc}
\usepackage{amsmath, amssymb}
\usepackage[margin=1in]{geometry}
\begin{document}

1. \textbf{Question}:

   For the following mean-field Hamiltonian, if we want to code and numerically solve the self-consistent equation, what is the dimension of the mean-field Hamiltonian matrix at a given momentum $k$? \\ \\Consider the mean-field Hamiltonian: $H = H_{\text{Kinetic}} + H_{\text{Hartree}} +H_{\text{Fock}}$, with each term defined below: \\$H_{\text{Kinetic}} = \sum_{s, k} E_s(k) c^\dagger_s(k) c_s(k)$, where $E_s(k)=\sum_{n} t_s(n) e^{-i k \cdot n}$  \\$H_{\text{Hartree}} = \frac{1}{N} \sum_{s, s'} \sum_{k_1, k_2} U(0) \langle c_s^\dagger(k_1) c_s(k_1) \rangle c_{s'}^\dagger(k_2) c_{s'}(k_2)$ \\$H_{\text{Fock}} = -\frac{1}{N} \sum_{s, s'} \sum_{k_1, k_2} U(k_1 - k_2) \langle c_s^\dagger(k_1) c_{s'}(k_1) \rangle c_{s'}^\dagger(k_2) c_s(k_2)$ \\$U(k) = \sum_{n} U_n e^{-i k \cdot n}$, where $U_0$ is the on-site interaction, $U_1$ is the nearest neighbor interaction, and $U_n$ is the n-th shell neighbor interaction. Here, we consider only the on-site interaction and the nearest neighbor interaction. \\Hopping parameters are $t_1 = 6$ meV for nearest-neighbor hopping and $t_2 = 1$ meV for next-nearest-neighbor hopping, which correspond to $t_s(n)$ in the kinetic energy term. $s = \{\uparrow, \downarrow\}$ is the spin index. \\The lattice is a two-dimensional triangular lattice.

   \textbf{Answer}:

   2 $\times$ 2 matrix. \\


\clearpage

2. \textbf{Question}:

   For the following mean-field Hamiltonian, what do you think is the possible order parameter we can write down for the system without breaking the translational symmetry? \\ \\Consider the mean-field Hamiltonian: $H = H_{\text{Kinetic}} + H_{\text{Hartree}} +H_{\text{Fock}}$, with each term defined below: \\$H_{\text{Kinetic}} = \sum_{s, k} E_s(k) c^\dagger_s(k) c_s(k)$, where $E_s(k)=\sum_{n} t_s(n) e^{-i k \cdot n}$  \\$H_{\text{Hartree}} = \frac{1}{N} \sum_{s, s'} \sum_{k_1, k_2} U(0) \langle c_s^\dagger(k_1) c_s(k_1) \rangle c_{s'}^\dagger(k_2) c_{s'}(k_2)$ \\$H_{\text{Fock}} = -\frac{1}{N} \sum_{s, s'} \sum_{k_1, k_2} U(k_1 - k_2) \langle c_s^\dagger(k_1) c_{s'}(k_1) \rangle c_{s'}^\dagger(k_2) c_s(k_2)$ \\$U(k) = \sum_{n} U_n e^{-i k \cdot n}$, where $U_0$ is the on-site interaction, $U_1$ is the nearest neighbor interaction, and $U_n$ is the n-th shell neighbor interaction. Here, we consider only the on-site interaction and the nearest neighbor interaction. \\Hopping parameters are $t_1 = 6$ meV for nearest-neighbor hopping and $t_2 = 1$ meV for next-nearest-neighbor hopping, which correspond to $t_s(n)$ in the kinetic energy term. $s = \{\uparrow, \downarrow\}$ is the spin index. \\The lattice is a two-dimensional triangular lattice.

   \textbf{Answer}:

   $\langle c^\dagger_{s} c_{s'} \rangle$ with $s,s'=\uparrow,\downarrow$. \\


\clearpage

3. \textbf{Question}:

   For the following mean-field Hamiltonian, if we want to code and solve the problem in the momentum space, what are the coordinates of the 6 Brillouin zone corners, $K$ point? Round to 2 decimal places. \\ \\Consider the mean-field Hamiltonian: $H = H_{\text{Kinetic}} + H_{\text{Hartree}} +H_{\text{Fock}}$, with each term defined below: \\$H_{\text{Kinetic}} = \sum_{s, k} E_s(k) c^\dagger_s(k) c_s(k)$, where $E_s(k)=\sum_{n} t_s(n) e^{-i k \cdot n}$  \\$H_{\text{Hartree}} = \frac{1}{N} \sum_{s, s'} \sum_{k_1, k_2} U(0) \langle c_s^\dagger(k_1) c_s(k_1) \rangle c_{s'}^\dagger(k_2) c_{s'}(k_2)$ \\$H_{\text{Fock}} = -\frac{1}{N} \sum_{s, s'} \sum_{k_1, k_2} U(k_1 - k_2) \langle c_s^\dagger(k_1) c_{s'}(k_1) \rangle c_{s'}^\dagger(k_2) c_s(k_2)$ \\$U(k) = \sum_{n} U_n e^{-i k \cdot n}$, where $U_0$ is the on-site interaction, $U_1$ is the nearest neighbor interaction, and $U_n$ is the n-th shell neighbor interaction. Here, we consider only the on-site interaction and the nearest neighbor interaction. \\Hopping parameters are $t_1 = 6$ meV for nearest-neighbor hopping and $t_2 = 1$ meV for next-nearest-neighbor hopping, which correspond to $t_s(n)$ in the kinetic energy term. $s = \{\uparrow, \downarrow\}$ is the spin index. \\The lattice is a two-dimensional triangular lattice.

   \textbf{Answer}:

   (0,  4.19) \\
(0, -4.19) \\
(3.62, 2.09) \\
(-3.62, 2.09) \\
(3.62, -2.09) \\
(-3.62, -2.09) \\


\clearpage

4. \textbf{Question}:

   For the following mean-field Hamiltonian, if we want to code and solve the problem in the momentum space, what are the coordinates of the center of Brillouin zone, $\Gamma$ point? Round to 2 decimal places. \\ \\Consider the mean-field Hamiltonian: $H = H_{\text{Kinetic}} + H_{\text{Hartree}} +H_{\text{Fock}}$, with each term defined below: \\$H_{\text{Kinetic}} = \sum_{s, k} E_s(k) c^\dagger_s(k) c_s(k)$, where $E_s(k)=\sum_{n} t_s(n) e^{-i k \cdot n}$  \\$H_{\text{Hartree}} = \frac{1}{N} \sum_{s, s'} \sum_{k_1, k_2} U(0) \langle c_s^\dagger(k_1) c_s(k_1) \rangle c_{s'}^\dagger(k_2) c_{s'}(k_2)$ \\$H_{\text{Fock}} = -\frac{1}{N} \sum_{s, s'} \sum_{k_1, k_2} U(k_1 - k_2) \langle c_s^\dagger(k_1) c_{s'}(k_1) \rangle c_{s'}^\dagger(k_2) c_s(k_2)$ \\$U(k) = \sum_{n} U_n e^{-i k \cdot n}$, where $U_0$ is the on-site interaction, $U_1$ is the nearest neighbor interaction, and $U_n$ is the n-th shell neighbor interaction. Here, we consider only the on-site interaction and the nearest neighbor interaction. \\Hopping parameters are $t_1 = 6$ meV for nearest-neighbor hopping and $t_2 = 1$ meV for next-nearest-neighbor hopping, which correspond to $t_s(n)$ in the kinetic energy term. $s = \{\uparrow, \downarrow\}$ is the spin index. \\The lattice is a two-dimensional triangular lattice.

   \textbf{Answer}:

   (0,0) \\


\clearpage

5. \textbf{Question}:

   For the following mean-field Hamiltonian, if we want to code and solve the noninteracting band structure, what are the energies at the center of Brillouin zone, $\Gamma$ point, for all bands? \\ \\Consider the mean-field Hamiltonian: $H = H_{\text{Kinetic}} + H_{\text{Hartree}} +H_{\text{Fock}}$, with each term defined below: \\$H_{\text{Kinetic}} = \sum_{s, k} E_s(k) c^\dagger_s(k) c_s(k)$, where $E_s(k)=\sum_{n} t_s(n) e^{-i k \cdot n}$  \\$H_{\text{Hartree}} = \frac{1}{N} \sum_{s, s'} \sum_{k_1, k_2} U(0) \langle c_s^\dagger(k_1) c_s(k_1) \rangle c_{s'}^\dagger(k_2) c_{s'}(k_2)$ \\$H_{\text{Fock}} = -\frac{1}{N} \sum_{s, s'} \sum_{k_1, k_2} U(k_1 - k_2) \langle c_s^\dagger(k_1) c_{s'}(k_1) \rangle c_{s'}^\dagger(k_2) c_s(k_2)$ \\$U(k) = \sum_{n} U_n e^{-i k \cdot n}$, where $U_0$ is the on-site interaction, $U_1$ is the nearest neighbor interaction, and $U_n$ is the n-th shell neighbor interaction. Here, we consider only the on-site interaction and the nearest neighbor interaction. \\Hopping parameters are $t_1 = 6$ meV for nearest-neighbor hopping and $t_2 = 1$ meV for next-nearest-neighbor hopping, which correspond to $t_s(n)$ in the kinetic energy term. $s = \{\uparrow, \downarrow\}$ is the spin index. \\The lattice is a two-dimensional triangular lattice.

   \textbf{Answer}:

   (48 meV, 48 meV) \\


\clearpage

6. \textbf{Question}:

   For the following mean-field Hamiltonian, if we want to code and solve the noninteracting band structure, what are the energies at the first Brillouin zone corners, $K$ point, for all bands? \\ \\Consider the mean-field Hamiltonian: $H = H_{\text{Kinetic}} + H_{\text{Hartree}} +H_{\text{Fock}}$, with each term defined below: \\$H_{\text{Kinetic}} = \sum_{s, k} E_s(k) c^\dagger_s(k) c_s(k)$, where $E_s(k)=\sum_{n} t_s(n) e^{-i k \cdot n}$  \\$H_{\text{Hartree}} = \frac{1}{N} \sum_{s, s'} \sum_{k_1, k_2} U(0) \langle c_s^\dagger(k_1) c_s(k_1) \rangle c_{s'}^\dagger(k_2) c_{s'}(k_2)$ \\$H_{\text{Fock}} = -\frac{1}{N} \sum_{s, s'} \sum_{k_1, k_2} U(k_1 - k_2) \langle c_s^\dagger(k_1) c_{s'}(k_1) \rangle c_{s'}^\dagger(k_2) c_s(k_2)$ \\$U(k) = \sum_{n} U_n e^{-i k \cdot n}$, where $U_0$ is the on-site interaction, $U_1$ is the nearest neighbor interaction, and $U_n$ is the n-th shell neighbor interaction. Here, we consider only the on-site interaction and the nearest neighbor interaction. \\Hopping parameters are $t_1 = 6$ meV for nearest-neighbor hopping and $t_2 = 1$ meV for next-nearest-neighbor hopping, which correspond to $t_s(n)$ in the kinetic energy term. $s = \{\uparrow, \downarrow\}$ is the spin index. \\The lattice is a two-dimensional triangular lattice.

   \textbf{Answer}:

   (-15 meV, -15 meV) \\


\clearpage

7. \textbf{Question}:

   For the following problem, derive the kinetic term of the Hamiltonian in second-quantized form ($H_{TB}$) for this triangular lattice system. Express your final answer as the complete tight-binding Hamiltonian, clearly showing the summations and indices. \\ \\Consider a triangular lattice system where the degrees of freedom are spin states associated with $+K$ and $-K$ valleys. The kinetic term of the Hamiltonian can be described by a tight-binding model with the following properties: \\Electrons hop between sites with amplitude $t_s(R_i - R_j)$. \\$s = \uparrow,\downarrow$ represents spin states associated with $+K$ and $-K$ valleys respectively. \\$R_i$ represents a site position in the triangular lattice. \\$c_{R_i,s}$ is the electron annihilation operator at site with position $R_i$ with spin $s$. \\$c^{\dagger}_{R_i,s}$ is the electron creation operator at site with position $R_i$ with spin $s$. \\The hopping process occurs between sites at positions $R_i$ and $R_j$. \\The summation should be taken over all spin states and all real space positions.

   \textbf{Answer}:

   $H_{TB} = -\sum_{R_i,R_j,s} t_s(R_i - R_j)c^\dagger_{R_i,s}c_{R_j,s}$ \\


\clearpage

8. \textbf{Question}:

   For the following problem, derive the interaction part of the Hamiltonian $H^{\text{int}}$ in the second-quantized form for this triangular lattice system. Express your final answer in terms of the number operators $n_s(R_i)$. \\ \\Consider a triangular lattice system with spin degrees of freedom associated with $+K$ and $-K$ valleys. We want to examine the interaction term of the Hamiltonian in the second-quantized form, which complements the kinetic term. \\The system has a density-density interaction between sites with the following properties: \\The interaction occurs between sites at positions $R_i$ and $R_j$ with interaction strength $U(R_i - R_j)$. \\$R_i$ and $R_j$ represent site positions in the triangular lattice. \\$c_{R_i,s}$ and $c^{\dagger}_{R_i,s}$ are the electron annihilation and creation operators. \\The summation should be taken over all spin states $s, s'$ and all real space positions. \\The number operator for electrons with spin $s$ at site position $R_i$ is defined as $n_s(R_i) = c^{\dagger}_{R_i,s}c_{R_i,s}$.

   \textbf{Answer}:

   $H^{\text{int}} = \frac{1}{2} \sum_{s,s'=\uparrow,\downarrow} \sum_{R_i,R_j} U(R_i - R_j)n_s(R_i)n_{s'}(R_j)$ \\


\clearpage

9. \textbf{Question}:

   For the following problem, express the noninteracting Hamiltonian $\hat{H}_{\text{Kinetic}}$ in terms of the second quantized operators $c_s^\dagger(k)$ and $c_s(k)$ in the momentum space. Simplify any summation indices if possible. \\ \\Consider a triangular lattice system with spin degrees of freedom associated with $+K$ and $-K$ valleys. We have a noninteracting Hamiltonian in the second-quantized form expressed in the real space basis as: \\$\hat{H}_{\text{Kinetic}} = \sum_{R_i,R_j} \sum_{s=\uparrow,\downarrow} t_s(R_i - R_j) c_s^\dagger(R_i)c_s(R_j)$ \\where: \\$c_s^\dagger(R_i)$ and $c_s(R_j)$ are creation and annihilation operators in real space \\$t_s(R_i - R_j)$ is the hopping parameter \\The summation is over all sites and spin states \\ \\The Fourier transformation from real space to momentum space is defined as: \\$c_s^\dagger(k) = \frac{1}{\sqrt{N}} \sum_i c_s^\dagger(R_i)e^{ik \cdot R_i}$ \\where $N$ is the number of unit cells in the real space.

   \textbf{Answer}:

   $\hat{H}_{\text{Kinetic}} = \sum_{s=\uparrow,\downarrow} \sum_k E_s(k)c_s^\dagger(k)c_s(k)$ \\
where $E_s(k) = \sum_n t_s(n)e^{-ik \cdot n}$ \\


\clearpage

10. \textbf{Question}:

   For the following problem, express the interaction Hamiltonian $\hat{H}^{\text{int}}$ in terms of second quantized operators $c_s^\dagger(k)$ and $c_s(k)$ in the momentum space. Simplify any summation indices if possible. \\ \\Consider a triangular lattice system with spin degrees of freedom associated with $+K$ and $-K$ valleys. We have an interaction Hamiltonian in the second-quantized form expressed in the real space basis as: \\$\hat{H}^{\text{int}} = \frac{1}{2} \sum_{R_i,R_j} \sum_{s,s'=\uparrow,\downarrow} U(R_i - R_j)n_s(R_i)n_{s'}(R_j)$ \\ \\where: \\$n_s(R_i) = c_s^\dagger(R_i)c_s(R_i)$ is the number operator at site $R_i$ with spin $s$ \\$U(R_i - R_j)$ is the interaction strength between sites $i$ and $j$ \\The summation is over all sites and spin states \\ \\The Fourier transformation from real space to momentum space is defined as: \\$c_s^\dagger(k) = \frac{1}{\sqrt{N}} \sum_i c_s^\dagger(R_i)e^{ik \cdot R_i}$ \\ \\where $k$ is defined within the first Brillouin zone and $N$ is the number of unit cells in the real space.

   \textbf{Answer}:

   $\hat{H}^{\text{int}} = \frac{1}{2N} \sum_{s,s'=\uparrow,\downarrow} \sum_{k,k',k'',k'''} U(k - k')c_s^\dagger(k)c_s(k')c_{s'}^\dagger(k'')c_{s'}(k''') \sum_G \delta(k-k'+k''-k''',G)$ \\
 \\
where $U(k) = \sum_n U(n)e^{-ik \cdot n}$ and $G$ is a reciprocal lattice vector. \\


\end{document}